\section{Heating and cooling in HII regions}

The code of any shared modules for question 2 is given by:
\lstinputlisting[firstnumber=1, firstline=1, lastline=71]{Q2.py}



\subsection{a}

The code used is given by:
\lstinputlisting[firstnumber=75, firstline=75, lastline=81]{Q2.py}

We calculate the equilibrium temperature by finding the root of equilibrium1 (as given in the code).
The functions are of order lower than quadratic, which is why we don't expect Brent's method
to work optimally. Therefore, we choose the linear false position method combined with bisection to overcome slow convergence in very non-linear regions.
This is implemented by using false position in general but switching to bisection whenever the bracket is not at least decreased by half in size, inspired
by the safeguards used in Brent's method. The algorithm stops (in this case) when an absolute error less than 0.1 K is reached. The results are given here:
\lstinputlisting[firstline=1, lastline=3]{output_Q2.txt}



\subsection{b}

The results for root finding of the function equilibrium2 (as given in the code) are given here. 
We find that the higher the density, the lower the temperature, which is as expected.
We further note that the error on the higher temperature is larger than the error on the lower temperatures, 
which is because the algorithm stops at a relative target accuracy of $10^{-10}$.

The code used is given by:
\lstinputlisting[firstnumber=85, firstline=85, lastline=93]{Q2.py}

The results:
\lstinputlisting[firstline=4, lastline=8]{output_Q2.txt}